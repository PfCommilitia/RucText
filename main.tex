\documentclass[12pt,punct=kaiming]{ctexart}
\usepackage{RucTextStyle}

\topic{鲁克思主义原理课程论文}
\title{当鲁克思遇见孔夫子}
\subtitle{对话形式的鲁克思主义地球化探究}
\author{鲁克思}
\school{鲁克思学院}
\major{鲁克思主义原理}
\grade{2198级}
\studentid{1145141919}
\advisor{孔夫子}
\date{2200年1月1日}

\begin{document}

\maketitle

\keywords{关键词1 \quad 关键词2 \quad 关键词3}
\enkeywords{Keyword1 \quad Keyword2 \quad Keyword3}

\newpage
\begin{abstract}
这是一段摘要。
\end{abstract}

\begin{enabstract}
This abstract.
\end{enabstract}
\newpage

\tableofcontents
\newpage

\makedoc

\section{引言}

引言是论文的开场白,应简要介绍问研究的目的和意义、前人相关成果、本文要解决的主要问题,以及解决方案的主要思路和预期效果。

正文是论文的主体。可分为若干章节,陈述相关工作调研、问题/需求分析、解决方案思路和内容、实验过程、结果和讨论等内容。应结构严谨、逻辑性强。

\section{相关工作}

BERT 是一个常用的预训练语言模型\cite{devlin2018bert}。\footnote{脚注:\url{https://huggingface.co/google-bert/bert-base-uncased}}

\section{方法}

\subsection{问题定义}

这是一段公式:
\begin{equation}
    E = mc^2
\end{equation}

\begin{itemize}
    \item 项目列表
    \item
\end{itemize}

\subsection{模型}

\begin{figure}[h]
    \centering
    \includegraphics[width=0.6\textwidth]{RucTextAssets/title.jpg}
    \caption{图标题在图下方}
    \label{fig:enter-label}
\end{figure}


\section{实验}

\subsection{实验设置}

\subsection{实验结果}

\begin{table}[h]
    \centering
    \caption{表标题在表上方}
    \begin{tabular}{@{}l|cccccc@{}}
    \toprule
    Model & dataset1 & dataset2 & dataset3 & dataset4 & dataset5 & \\
    \midrule
    1 & 100.0 & & & & & \\
    2 & 99.0 & & & & & \\
    3 & 98.0 & & & & & \\
    4 & 97.0 & & & & & \\
    \bottomrule
    \end{tabular}
    \label{tab:my_label}
\end{table}

\section{结论}

\newpage
\addcontentsline{toc}{section}{参考文献}
\bibliographystyle{gbt7714-numerical}
\bibliography{references.bib}

\newpage
\addcontentsline{toc}{section}{附录}
\section*{附录}
\appendix
\subsection*{附录1}

\end{document}
